\makeatletter
\def\input@path{{../sty/}}
\makeatother

\documentclass{article}
\usepackage{cpp_advice}

\begin{document}
\title{What's Const For}
\author{Ashwin Menon}
\date{\today}
\maketitle

What good are \emph{const} variables? A good compiler would not allocate storage for
them, and an enumeration achieves the same effect. According to Bjarne Stroustrupp
in \emph{The C\texttt{++} Programming Language},

\begin{displayquote}
const’s primary role is to specify immutability in interfaces.
\end{displayquote}

Making an object \emph{const} does not specify how the compiler allocates storage
for that object; instead, it places limitations on the usage of that object.
Consider the \emph{strcpy} function:

\begin{lstlisting}
char *strcpy(char *dest, const char *src);
\end{lstlisting}

The source is a pointer to a \emph{const char}, which is how the designer of 
\emph{strcpy} tells the compiler that \emph{strcpy} will not modify the source 
string. This allows the compiler to make optimizations where possible.

Consider the \emph{push\_back} method of \emph{vector}:

\begin{lstlisting}
void push_back(const T& value);
\end{lstlisting}

Not only is the argument passed by reference to avoid unnecessary copying, but
the reference is also a \emph{const} which tells the compiler that \emph{push\_back}
promises not to modify its argument. Accidental argument modifications within 
\emph{push\_back} will therefore be flagged as an error by the compiler.

\end{document}