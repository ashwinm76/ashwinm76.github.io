\makeatletter
\def\input@path{{../sty/}}
\makeatother

\documentclass{article}
\usepackage{cpp_advice}

\begin{document}
\title{Be Careful Using Chars}
\author{Ashwin Menon}
\date{\today}
\maketitle

You might think that the \emph{char} would be the perfect data type to represent
byte sized variables, especially in an embedded system where storage space is at
a premium. However, there are serious pitfalls with this, not least of which is
the fact that it is implementation defined whether a \emph{char} is a signed or an
unsigned value. Consider this:

\begin{lstlisting}
char c = 255; // Is c signed or unsigned?
int i  = c;   // Is i -1 or 255?
\end{lstlisting}

The moment you try to store the \emph{char} in a larger numeric value, then the
larger value becomes implementation defined. You might think you could get around
this by using \emph{unsigned char} or \emph{signed char} variables to make the
signedness explicit, and therefore under your control. But that has pitfalls as 
well. Consider this:

\begin{lstlisting}
char ch;
signed char schar;
unsigned char uchar;

char *p_char = &uchar;           // error
signed char *p_schar = p_char;   // error
unsigned char *p_uchar = p_char; // error
p_schar = p_uchar;               // error
\end{lstlisting}

All four pointer assignments are illegal in C\texttt{++} and will be flagged as invalid
conversions. This makes it a problem when you try to pass a \emph{signed char}
or an \emph{unsigned char} pointer to any of the standard library string 
functions, all of which take \emph{char*} arguments. Consider:

\begin{lstlisting}
const char *str = "hello";
const signed char *s_str = "goodbye"; // error
const unsigned char *u_str = "ringo"; // error

int i = strlen(str);
int j = strlen(s_str); // error
int k = strlen(u_str); // error
\end{lstlisting}

The second and third string assignments are errors, because you can’t convert a
\emph{char*} to a \emph{signed char*} or an \emph{unsigned char*}. For the same
reason (working in reverse), the second and third calls to \emph{strlen} are
errors, since \emph{strlen} only accepts \emph{char*} arguments.

You can avoid all this bother by simply using \emph{char} everywhere instead of
the \emph{signed} or \emph{unsigned} variant, and by also making sure that you
never store negative or too large values in these \emph{char}'s. This makes it
perfectly fine to use \emph{char}'s to store ASCII strings, since ASCII 
characters use only the lower 127 values, and therefore can never be sign 
extended to negative values.

If you want to store 8-bit numeric values that don’t represent ASCII characters
(i.e. they really are numbers) then use one of the standard integer types, like
\emph{uint8\_t} or \emph{int8\_t}.

\end{document}